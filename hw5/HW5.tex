\documentclass{article}
\usepackage{listings}
\usepackage{amsmath}
\usepackage{amsfonts}
\usepackage{blindtext}
\usepackage{geometry}
\geometry{verbose,tmargin=3cm,bmargin=3cm,lmargin=3cm,rmargin=3cm}

\lstset{
    frame=single,
}

\title{Crypto - HW 5}
\author{Hagai Ben Yehuda, ID num: 305237000\\ Jonathan Bauch, ID num: 204761233}
\date{}
\renewcommand{\thesubsection}{\thesection.\alph{subsection}}
\interlinepenalty 10000
\begin{document}
  \maketitle

\section{1}
\subsection{}%1.a
ndeed, suppose we have the signature $\sigma_m$ of $m\in\{1,..., n\}$, then we can obtain the signature of any $k \in\{1,...,n\}$ that satisfies $k < m$ by simply setting $\sigma_k = f^{m-k}(\sigma_m)$, then we have that 
\[f^{k}(\sigma_k) = f^{k}(f^{m-k}(\sigma_m)) = f^{m}(\sigma_m) = y\]
With the last equality is due to the assumption that $\sigma_m$ is the correct signature for $m$.
Thus this is not a one time secure signature scheme.

\subsection{}%1.b
First we show that $f^k$ is also a permutation, we do so using induction:\\
For the base case where $k = 1$ this is true from the definition of $f$.
Now assume that $f^{k-1}$ is a permutation, let $x\in\{0,1\}^n$ then $x$ has a source under $f$ (as $f$ is a permutation), namely there exists $y\in\{0,1\}^n$ such that $f(y) = x$, from the induction assumption we know that $f^{k-1}$ is a permutation and thus $y$ has a source $z\in\{0,1\}^n$ such that $f^{k-1}(z) = y$, thus $f(f^{k-1}(z)) = f(y) = x$, thus $x$ has a source under $f^k$.
Since $\{0,1\}^n$ is finite and since we have showed that $f^k$ is on-to, we have that $f$ is also one to one, Hence $f$ is a permutation.
Now assume that $f^k$ is not a one-way then there is a polynomial time algorithm $A$ that satisfies: 
\[\Pr_{x\leftarrow \{0,1\}^n}(A(f^k(x) = x) > \epsilon\]
We define an algorithm $A'$ that does the following on input x calculates $f^k(x)$ and feeds it to $A$. $A'$ is polynomial since $A$ and $k$ are polynomial, also note that:
\[\Pr_{x\leftarrow \{0,1\}^n}(A'(x) = x) = \Pr_{x\leftarrow \{0,1\}^n}(A(f^k(x) = x) > \epsilon\]
Leading the a contradiction to the assumption that $f$ is a OWP, hence no such $A$ exists and $f^k$ is also a OWP.
\subsection{}%1.c
ndeed, assuming that there is some polynomial algorithm $A$ such that for some $m\in\{1,...n\}$, $\sigma_m = f^{n-m}(m)$ and $m'>m$, $A$ satisfies:
\[\Pr_{x\leftarrow \{0,1\}^n}(A(\sigma_m) = \sigma(m') = f^{n-m'}(x)) > \epsilon\]
We construct a polynomial algorithm $A'$ that inverts $f$ with the same probability, on input $f(w)$ $A'$ will do the following:
\begin{itemize}
\item Set $k = m' - m$.
\item Set $\sigma_m = f^{k-1}(f(w)) = f^k(w)$.
\item Execute $A(\sigma_m)$ and return its result.
\end{itemize}
Then
\begin{align*}
\Pr_{w\leftarrow \{0,1\}^n}(A'(f(w)) = w) &= \Pr_{x\leftarrow \{0,1\}^n}(A'(f^{n-m}(x)) = w) \\
&= \Pr_{x\leftarrow \{0,1\}^n}(A(\sigma_m) = w)\\
&= \Pr_{x\leftarrow \{0,1\}^n}(A(\sigma_m) =  f^{-k}(\sigma_m)))\\
&= \Pr_{x\leftarrow \{0,1\}^n}(A(\sigma_m)  = \sigma_{m'}) > \epsilon
\end{align*}
The first equality is due to the fact that given a random $w$, the probability for any $x\leftarrow \{0,1\}^n$ to be its $k$'th source is equal for every $x$ as we assume that $f$ was chosen uniformly form the random permutation functions.
This is obviously a contradiction to the assumption that $f$ is a OWF, showing no such algorithm $A$ exists.
\subsection{}%1.d
//////////// need to do this!!!!!!!!!! ////////////
\section{}%2
Let $\tilde{f}$ be some one way function, and define $f$ such that $f(xl) = f(x0)$ (with $l$ being a single bit).
Suppose that $f$ is not a one way function, then there is an algorithm $A$ such that 
\[\Pr_{x\leftarrow \{0,1\}^n}(A(f(x)) = x) > \epsilon\]
Construct an algorithm $A'$ that on input $\tilde{f}(x)$ executes $A$ to receive $yl$ and returns $y0$.
Then we have:
\begin{align*}
\Pr_{x\leftarrow \{0,1\}^n}(A'(\tilde{f}(x)) = x) &\geq \frac{1}{2}\Pr_{x\leftarrow \{0,1\}^{n-1}}(A'(\tilde{f}(x0)) = x0)\\
&= \frac{1}{2}\Pr_{x\leftarrow \{0,1\}^{n}}(A(f(x)) = x)\\
&>\frac{\epsilon}{2}
\end{align*}
The equality (in the second line) is correct because $A'$ will produce a valid result whenever $A$ is a able to find the source of a message in $\{0,1\}^{n}$ because if it finds a source then we know that zeroing the last bit of the result will be the source of the original function.
Now that we have constructed $f$ and have shown that it is a one way function, then if we use Lamport’s scheme with $f$ and have a valid signature $(x_{m_1, 1}, ..., x_{m_n, n})$ of the message $(m_1, ... , m_2)$ then we know that if $x_{m_1, 1} = xl$ then  $(x\overline{l}, ..., x_{m_n, n})$ is also a signature for the same message from the construction of $f$, since $f(xl) = f(x\overline{l})$, showing that using $f$ Lamport’s scheme is not a strong one time signature scheme.

\section{}%3

\section{}%4
\subsection{}%4.a

\subsection{}%4.b

\section{}%5

\section{}%6
\subsection{}%6.a

\subsection{}%6.b

\end{document}