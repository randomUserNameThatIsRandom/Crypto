\documentclass{article}
\usepackage{listings}
\usepackage{amsmath}
\usepackage{amsfonts}
\usepackage{blindtext}
\usepackage{geometry}
\usepackage{enumerate}
\geometry{verbose,tmargin=3cm,bmargin=3cm,lmargin=3cm,rmargin=3cm}
\title{Crypto - HW 2}
\author{Hagai Ben Yehuda, ID num: 305237000\\ Jonathan Bauch, ID num: 204761233}
\date{}
\renewcommand{\thesubsection}{\thesection.\alph{subsection}}
\begin{document}
  \maketitle

\section{}

\subsection{}
Let $A$ be an efficient algorithm such that given $h = g^x$, $A$ outputs $x$ with probability $\epsilon$.\\
Define: 
\[A'(g^x, g^y) = (g^x)^{A(y)}\]
Note that if $A(g^y) = y$ then 
\[A'(g^x, g^y) = (g^x)^{A(y)} = (g^x)^y = g^{xy}\]
Thus
\begin{align*}
&\Pr(A' \mbox{ is correct}) = \Pr(A'(g^x, g^y) = g^{xy} )\\
&  = \Pr((g^x)^{A(y)} = g^{xy})=
  \Pr(A(g^y) = y) = \epsilon
\end{align*}

Since $A$ is efficient, $A'$ is an efficient algorithm with the required property.

\subsection{}
Let $A$ be an efficient algorithm as described in the question.
Construct $A'$ as follows:

\[
A'(g^x, g^y, g^z) = 
\begin{cases}
	 1 &\mbox{if }  g^z = A(g^x, g^y) \\ 
	 0 &\mbox{else}
\end{cases} 
\]

Then we have:
\[ \Pr(A'(g^x, g^y, g^{xy}) = 1 | (x,y) \in  \mathbb{Z}_{|G|} \times \mathbb{Z}_{|G|}) = \Pr(A(g^x, g^y) = g^{xy}) = \epsilon\]

On the other hand we have:
\[  \Pr(A'(g^x, g^y, g^{z}) = 1 | (x,y,z) \in  \mathbb{Z}_{|G|} \times \mathbb{Z}_{|G|} \times \mathbb{Z}_{|G|}) = \Pr(A(g^x, g^y) = g^{z}) = \frac{1}{|G|} \]
Where the last equality stems from the fact that for every $x, y$ there is a single value $z\in \mathbb{Z}_{|G|}$ (as $g$ is a generator) such that 
$A(g^x, g^y) = g^z$ and since $z$ is chosen uniformly from $\mathbb{Z}_{|G|}$ the equality follows.
Note that even if $A$ is not deterministic this equality holds as:
\begin{align*}
&\Pr(A(g^x, g^y) = g^{z}) =\\
& \sum_{a \in \mathbb{Z}_{|G|}} \Pr(A(g^x, g^y)= g^z | A(g^x, g^y) = g^a) \Pr(A(g^x,g^y) = g^a) =\\
& \sum_{a \in \mathbb{Z}_{|G|}} \Pr(z=a) \Pr(A(g^x, g^y) = g^a) =\\
& \frac{1}{|G|} \sum_{a \in \mathbb{Z}_{|G|}} \Pr(A(g^x, g^y) = g^a) = \frac{1}{|G|}
\end{align*}


Thus we obtain:
\begin{align*}
&|\Pr_{(x,y) \leftarrow  \mathbb{Z}_{|G|} \times \mathbb{Z}_{|G|}}(A'(g^x, g^y, g^{xy}) = 1)  &\\
&- \Pr_{ (x,y,z) \leftarrow  \mathbb{Z}_{|G|} \times \mathbb{Z}_{|G|} \times \mathbb{Z}_{|G|}}(A'(g^x, g^y, g^{z}) = 1)| =\\
&= \epsilon - \frac{1}{|G|}
\end{align*}

Thus $A'$ is a distinguisher with the requested properties.
\section{}
Assume such an efficient algorithm $A$ exists, construct:
\[A'(v) = 
\begin{cases}
	 1 &\mbox{if }  f(A(v)) = v \\ 
	 0 &\mbox{else}
\end{cases} \]
Note that $A'$ is also efficient.

Now: 

\begin{align*}
&\Pr_{ v \leftarrow f(U_n)}(A'(v) = 1) = \Pr_{ v \leftarrow f(U_n)}(f(A(v)) = v) \\
&= \Pr_{ v \leftarrow f(U_n)}(A(v) \in f^{-1}(v)) \geq \epsilon
\end{align*}

On the other hand:
\begin{align*}
&\Pr_{v \leftarrow U_{n+s}}(A'(v) = 1) = \Pr_{v \leftarrow U_{n+s}}(f(A(v)) = v) = \\
&= \Pr_{v \leftarrow U_{n+s}}(A(v) \in f^{-1}(v))
\end{align*}

But note that $\Pr(f^{-1}(v) \neq \emptyset) \leq  \frac{1}{•2^s}$
Since  there are $2^{n+s}$ elements in $U_{n+s}$  and $f$ is a function, $f:\{0,1\}^n  \rightarrow \{0,1\}^{n+s}$
there are at most $2^n$ elements in $f$'s image, thus the probability that $v \in U_{n+s}$ has an origin according to f,
is at most $\frac{2^n}{2^{n+s}} = \frac{1}{2^s}$ (this bound is of-course achieved if $f$ is one-to-one). thus $\Pr(f^{-1}(v) \neq \emptyset) \leq  \frac{1}{•2^s}$ and hence : $\Pr_{v \leftarrow U_{n+s}}(A(v) \in f^{-1}(v)) \leq \frac{1}{2^s}$.\\
This yields that: 
\[|\Pr_{v \leftarrow f(U_n)}(A'(v) = 1) - \Pr_{ v \leftarrow U_{n+s}}(A'(v) = 1) | \geq \epsilon - \frac{1}{2^s}\]
As requested.

\section{}
Assume $G:\{0,1\}^n \rightarrow \{0,1\}^l$ is a PRG and assume for contradiction that $\varepsilon$ is not indistinguishable in the presence of an eavesdropper. Then there exists a PPT adversary A such that $\Pr(A \mbox{ Wins}) > \frac{1}{2} + \epsilon$.\\
We shall now construct a PPT adversary D that distinguishes $G(U_n)$ from $U_l$.
On input v D will do the following:
\begin{itemize}
\item Execute $A$ and obtain two messages, $m_0, m_1$.
\item Draw $b \leftarrow \{0, 1\}$.
\item Supply A with $m_b \oplus v$, if $A$ was correct return $1$ else return $0$.
\end{itemize}
We now show that D is a distinguisher between the distributions $G(U_n)$ and $U_l$ and thus arrive at a contradiction to the fact that $G$ is a PRG.

\begin{align*}
&\Pr_{v \leftarrow G(U_n)}(D(v) = 1) = \Pr_{v \leftarrow G(U_n)}(A(m_b \oplus v) = b) =\\
&= \Pr(A \mbox{ Wins}) > \frac{1}{2} + \epsilon
\end{align*}
On the other hand 

\[\Pr_{ v \leftarrow U_l}(D(v) = 1) = \Pr_{ v \leftarrow U_l}(A(m_b \oplus v) = \overline{b}) = \frac{1}{2}\]

Assume for the sake of contradiction that the last equality is incorrect, then there are two possibilities:
\begin{itemize}
\item $\Pr_{ v \leftarrow U_l}(A(m_b \oplus v) = \overline{b} ) > \frac{1}{2}$ : In this case if we define $f(v) = \overline{A(v)}$ then 
$\Pr_{ v \leftarrow U_l}(f(m_b \oplus v) = b) > \frac{1}{2}$ and thus F is a PPT adversary that show that one time pad is not adversarial indistinguishable
which is a contradiction to a theorem we proved.
\item $\Pr_{ v \leftarrow U_l}(A(m_b \oplus v) = \overline{b}) < \frac{1}{2}$ then  $\Pr_{ v \leftarrow U_l}(A(m_b \oplus v) = b) > \frac{1}{2}$ and thus A is a PPT adversary that show that one time pad is not adversarial indistinguishable.
\end{itemize}

hence, indeed $\Pr_{ v \leftarrow U_l}(D(v) = 1) = \frac{1}{2}$.
Thus:
\[|\Pr_{v \leftarrow G(U_n)}(D(v) = 1) - \Pr_{ v \leftarrow U_l}(D(v) = 1)| > \epsilon\]
which is a contradiction to the fact that $G$ is a PRG, and thus $\varepsilon$ is indeed indistinguishable in the presence of an eavesdropper.

\section{}
We will construct a PPT adversary D that distinguishes between oracle access to a function from $F$ and oracle access to a random function.
We define D for an input f:
\begin{itemize}
\item Query the value of $f$ for the string $0^n$ and save that result as $A_1$
\item Query the value of $f$ for the string $10^{n-1}$ and save that result as $A_2$
\item If $A_1$ and $A_2$ are identical except for the first bit (for which $A_1[0] = \overline{A_2[0]} $) return 1, else return 0.
\end{itemize}

We now prove that D is in-fact a PPT distinguisher.\\
 First note it in-fact runs in polynomial time as the oracles perform in polytime and $n$ bit comparison is also done in polytime. Now note that:
 
\[\Pr_{f \leftarrow F}(D(f) = 1) = 1 \]
This equality holds since if $f \in F$ then $f(0^n) = G(k) \oplus 0^n = G(k)$ for some $k \in \{0,1\}^n$ and 
$f(10^{n-1}) = G(k) \oplus 10^{n-1} = \overline{G(k)[0]}G(k)[1... n-1]$  for the same $k$ and thus $D$ will return $1$ for f.
We denote $URF$ the uniform distribution over the functions $f:\{0,1\}^n \rightarrow \{0,1\}^n$.
\[\Pr_{f \leftarrow URF}(D(f) = 1) = Pr_{f \leftarrow URF}(f(0^n) = \overline{f(10^{n-1})[0]} f(10^{n-1})[1...n-1]) = \frac{1}{2^{n}}\]
Where the last equality is true since $f$ was chosen in random and thus the probability of it having a certain value for input $x$ is identical for each value and there are $2^n$ possible values, hence:
\[|\Pr_{f \leftarrow F}(D(f) = 1) - \Pr_{f \leftarrow URF}(D(f) = 1)| = 1- \frac{1}{2^n}\]

And thus $F$ is indeed not a PRF.

\section{}
\subsection{}
This claim is incorrect.\\
Assume for the purpose of contradiction that such a hard core predicate exists, we denote the hard-core predicate for an OWF $f:\{0,1\}^n \rightarrow \{0,1\}^l$ by $HC_f$
Note that $HC_f$ operates on $f$'s input.
We now define a new function $g:\{0,1\}^n \rightarrow \{0,1\}^{l+1}$ as $g(x) = (f(x), HC_f(x))$.
Note that since $HC_f$ only operates on the input and is a generic HCP, $HC_f = HC_g$, thus if $g$ is indeed an OWF $HC_f$ is a hard-core predicate for $g$, but given $g(x)$, $HC_f(x)$ is fully known (it is simply the last bit of $g(x)$) thus $HC_f$ is not a hard core predicate for $g$, it now remains to show that $g$ is in-fact a OWF which will contradict the assumption that a generic hard-core predicate exists.\\
We will show that if $g$ is not a OWF then $f$ is not a OWF in contradiction to how we chose it.

Assume that $g$ is not a OWF, then there exists an algorithm $A$, that inverts $g$ with probability $\geq \epsilon$.\\
We define $D$, and show that it inverts $f$ with probability $\epsilon$ (which will contradict the fact that $f$ is a OWF).
On input $y \in \{0,1\}^n$ we define $D$ as follows:
\begin{itemize}
\item Execute $A$ on $y0$, denote the result $x$. If $f(x)$ = $y$ return $y$.
\item Execute $A$ on $y1$, denote the result $x$. If $f(x)$ = $y$ return $y$.
\item return $0^n$
\end{itemize}

\begin{align*}
\Pr(D \mbox{ inverts } f(x)) &\geq \Pr(A \mbox{ inverts } f(x)0 ) + \Pr(A \mbox{ inverts } f(x)1 )\\ 
&\geq \Pr(A \mbox{ inverts } g(x) ) \geq \epsilon
\end{align*}
Where the second to last inequality is correct since either $f(x)0$ or $f(x)1$ is exactly $g(x)$.
Thus we arrive at a contradiction to the fact that $f$ is a OWF.

\subsection{}
This claim is incorrect.
Define $f(x) = 0^n$ for all $x \in \{0,1\}^n$ and define a HCP for $f$: $HC_f(x) = \mbox{x's first bit}$.
We first show that this is indeed a HCP:\\
Note that for all $x \in \{0,1\}^n$ $f(x)$ assumes the same value, thus if $A$ is a PPT adversary that proves $HC_f$ is not a HCP for $f$, it must predict the value of $x$'s 
first bit with no information at all (as $f(x)$ is constant) in other words $A$ predicts a random bit with probability greater then $\frac{1}{2}$, which is a contradiction to the bit being random and thus no such $A$ exists.\\
Note that since $f$ is a constant function, it can be inverted with probability $1$ as an adversary which always returns $0^n$ is always successful in finding an inverse to $f(x)$ ($f$ is constant). Thus $f$ is not a OWF but it does have a HCP as shown.

\section{}
\subsection{}
To show $QR\leq\mathbb{Z}_{p}^{*}$ it is enough to show (i) $1\in QR$,
(ii) closure under multiplication, and (iii) closure under inversion.

\begin{enumerate}[i]
\item Indeed $1\in QR$, because $1\equiv1^{2}\pmod p$
\item Let $s_{1},s_{2}\in QR$. Then are $r_{1},r_{2}\in\mathbb{Z}_{p}^{*}$
s.t. $s_{1}\equiv r_{1}^{2}\pmod p$, $s_{2}\equiv r_{2}^{2}\pmod p$.
Then follows:
\[
s_{1}s_{2}\equiv r_{1}^{2}r_{2}^{2}=\left(r_{1}r_{2}\right)^{2}\pmod p
\]
 and $r_{1}r_{2}\in\mathbb{Z}_{p}^{*}$ because $\mathbb{Z}_{p}^{*}$
is a group. Therefore $s_{1}s_{2}\in QR$ by definition.
\item Let $s\in\mathbb{Z}_{p}^{*}$. Then there is $r\in\mathbb{Z}_{p}^{*}$
s.t. $s\equiv r^{2}\pmod p$. Therefore:
\[
s^{-1}\equiv\left(r^{2}\right)^{-1}=\left(r^{-1}\right)^{2}\pmod p
\]
and $r^{-1}\in\mathbb{Z}_{p}^{*}$, therefore$s^{-1}\in QR$.
\end{enumerate}

\subsection{}
Let $g\in\mathbb{Z}_{p}^{*}$, $\mathbb{Z}_{p}^{*}=\left\langle g\right\rangle $.
Assume by contradiction that $g\in QR$. Then there is an $r\in\mathbb{Z}_{p}^{*}$
s.t. $g\equiv r^{2}\pmod p$. Because $g$ generates $\mathbb{Z}_{p}^{*}$
there exists $i\in\mathbb{Z}$ s.t. $r\equiv g^{i}\pmod p$ . Therefore
$g\equiv g^{2i}\pmod p\iff g^{2i-1}\equiv1\pmod p$. Therefore $2i-1\vert o\left(g\right)=\varphi\left(p\right)=p-1$.
Note that for prime $p>2$ we know $p$ is odd, therefore $p-1$ is
even. On the other hand $2i-1$ is odd, therefore we get a contradiction
(that odd divides even), meaning $g\notin QR$. For the edge case
where $p=2$, $\mathbb{Z}_{p}^{*}$ is the trivial group (of 1 element),
and in this case the claim is not true (because $\mathbb{Z}_{p}^{*}=QR=\left\langle 1\right\rangle $).
From now on we will assume $p>2$.

\subsection{}
Let $g\in\mathbb{Z}_{p}^{*}$, $\mathbb{Z}_{p}^{*}=\left\langle g\right\rangle $.

\begin{enumerate}[i]
\item Let $a\in\mathbb{Z}_{p}^{*}$. Assume $a\in QR$. Then there exists
$r\in\mathbb{Z}_{p}^{*}$ s.t. $a\equiv r^{2}\pmod p$. Because $g$
generates $\mathbb{Z}_{p}^{*}$, there exists a $k\in\mathbb{Z}$
s.t. $r\equiv g^{k}\pmod p$. Therefore $a\equiv r^{2}\equiv\left(g^{k}\right)^{2}=g^{2k}\pmod p$.
\item Let $a\in\mathbb{Z}_{p}^{*}$. Assume that $a\equiv g^{2k}\pmod p$
for some k. Then $a\equiv\left(g^{k}\right)^{2}\pmod p$, $g^{k}\in\mathbb{Z}_{p}^{*}$,
therefore by definition $a\in QR.$
\end{enumerate}


\subsection{}
Let $a\in\mathbb{Z}_{p}^{*}$, $\mathbb{Z}_{p}^{*}=\left\langle g\right\rangle $.

\begin{enumerate}[i]
\item Assume $a\in QR$. Then by (c) there is a $k\in\mathbb{Z}$ s.t. $a\equiv g^{2k}\pmod p$.
Therefore
\[
a^{\frac{p-1}{2}}\equiv\left(g^{2k}\right)^{\frac{p-1}{2}}=\left(g^{p-1}\right)^{k}\equiv1^{k}=1\pmod p
\]
(Note that $g^{p-1}\equiv1$ because $o\left(g\right)=p-1$)
\item Assume $a^{\frac{p-1}{2}}\equiv1\pmod p$. There is an $i\in\mathbb{Z}$
s.t. $a\equiv g^{i}\pmod p$. Therefore
\[
1\equiv\left(g^{i}\right)^{\frac{p-1}{2}}=\left(g^{\frac{p-1}{2}}\right)^{i}\overset{*}{=}\left(-1\right)^{i}\pmod p
\]
Thus $i$ is even (again assuming $p>2$). Denote $i=2k$, and now
by (c) we get that $a\in QR$.\\
{*} - This can be explained as follows: $g^{\frac{p-1}{2}}\not\equiv1$
because $o(g)=p-1>\frac{p-1}{2}$, and $\left(g^{\frac{p-1}{2}}\right)^{2}=g^{p-1}\equiv1$.
Therefore necessarily $g^{\frac{p-1}{2}}\equiv-1$, because $\pm1$
are the only square roots of $1\pmod p$ . (There are no other roots
because $x^{2}-1$ as at most $2$ roots)
\end{enumerate}

\subsection{}
Denote $a=g^{x}\mod p$. Then x is even $\iff$ there is a $k\in\mathbb{Z}$
s.t. $x=2k$ $\iff$ $a\in QR$ (by c) $\iff$ $a^{\frac{p-1}{2}}\equiv1\pmod p$
(by d).\\
Therefore given $f\left(x\right)=g^{x}\mod p$ we can compute $b:=\left(f\left(x\right)\right)^{\frac{p-1}{2}}\mod p$.
If $b=1$ then necessarily x is even, i.e. $parity(x)=0$. Otherwise
(if $b=0)$ x is odd, i.e. $parity(x)=1$.\\
$a^{n}\mod p$ can be computed efficiently as follows: compute values
$a^{2^{k}}$ iteratively by squaring (mod p), until $2^{k}\ge n$
(note that we don't have to store $a^{2^{k}}$ in memory, as we compute
(mod p)). Then according to the binary representation of $n$, multiply
these values for which the k'th bit of $n$ is 1. The whole process
involves $O\left(\log_{2}\left(n\right)\right)$ multiplications mod
p, which is linear in the number of bits of $n$.

\section{}

\lstinputlisting[language=Python, caption=Code]{ex1_q7.py}

\begin{lstlisting}[caption=Execution output]
max cycle length: 131071
Test a
first 30 bits:  [1, 0, 0, 0, 0, 0, 0, 0, 0, 0, 0, 0, 0, 0, 0,
0, 0, 1, 0, 0, 0, 0, 0, 0, 0, 0, 0, 0, 0, 1]
cycle length = 131071
precentage of max cycle: 100%
Test b
first 30 bits:  [1, 0, 0, 0, 0, 0, 0, 0, 0, 0, 0, 0, 0, 0, 0,
0, 0, 1, 0, 0, 0, 0, 0, 0, 0, 0, 1, 0, 0, 0]
cycle length = 35805
precentage of max cycle: 27%
\end{lstlisting}

\subsection{}
As seen from the execution output, LFSR1 has the maximum cycle length of $2^{17}-1$

\subsection{}
LFSR2 with the given input state has a cycle of $35805$ steps, which is only 27\% of the maximum $2^{17}-1$.

\end{document}