\documentclass{article}
\usepackage{listings}
\usepackage{amsmath}
\usepackage{amsfonts}
\usepackage{blindtext}
\usepackage{geometry}
\geometry{verbose,tmargin=3cm,bmargin=3cm,lmargin=3cm,rmargin=3cm}
\title{Crypto -HW 4}
\author{Hagai Ben Yehuda, ID num: 305237000\\ Jonathan Bauch, ID num: 204761233}
\date{}
\renewcommand{\thesubsection}{\thesection.\alph{subsection}}
\interlinepenalty 10000
\begin{document}
  \maketitle

\section{}

\section{•}
We construct a randomized algorithm $A'$ that operates on input $m = pq$ as follows:
\begin{enumerate}
\item Draw $y\in Z_{m}^{*}$ uniformly (we do that by drawing from $\{1,...,m-1\}$ and making sure $\gcd(y,m)$ is zero, if it isn't we can factor m using $y$). .
\item Execute $A$ on input $y^2 \pmod m $ and set $x$ to be its result (note that since $y^2 \pmod m $ is a quadratic residue we will get a number and not  "go catch a Stellagama stellio"). 
\item If $x = \pm y \pmod m$ and this step was executed less then $c$ times ($c$ being a constant positive integer that will affect the probability of success) go to step one if this step was executed $c$ times, return $0$.
\item Calculate $w = x  y^{-1} \pmod m$.
\item Set $k = w+1 \pmod m$ 
\item Set $z = \frac{k}{2}$.
\item set $q = gcd(z, m)$ and return $(q, \frac{m}{q})$
\end{enumerate} 
We shall now prove that $A'$ runs in $O(t(n))$ and finds a factorization for $m$ with probability $ 1- \frac{1}{2^c}$.
First note that m executes steps 1 through 3 at most $c$ time (from the restriction in step 3) and each steps takes $O(t(n))$ steps.
In addition for each execution $A'$ passes step $3$ with probability $\frac{1}{2}$, that is because $y^2$ has four roots in $Z_{m}^{*}$, and only two of them are $\pm y$, since $y$ was chosen uniformly, the probability that the root that $A$ returns for $y^2$ is $ \pm y$  is  $\frac{2}{4} = \frac{1}{2}$.
Now we prove that if $A'$ passes step 3 it returns a correct factorization.

From the CRT we can write $x = wy$ with 
\[w = a_1(q^{-1} \pmod p)q + a_2(p^{-1} \pmod q)p\]
and $a_i \in \{\pm 1\}$ (this is because as stated in the lecture, if $x$ is a root of $y$ then it can be written as $ly$ with $l$ being a root of $1$ in $m$), since we chose x such that $x \neq \pm y$, we know that $a_1 \neq a_2$.\\
Assume without loss of generality that $a_1 = 1$ and  $a_2 = -1$, thus we have ( $w$ is from step 4)
\[w = (q^{-1} \pmod p)q - (p^{-1} \pmod q)p \]
Note that from fermat's little theorem we have 
\[(q^{-1} \pmod p)= q^{p-2} + cp\]
\[(p^{-1} \pmod q)= p^{q-2} + rq\]
Thus 
\[w = q^{p-1} - p^{q-1} \pmod {pq}\]
Note that 
\[q^{p-1} + p^{q-1} = 1 \pmod p\]
and
\[q^{p-1} + p^{q-1}  = 1 \pmod q\]
Hence
\[q^{p-1} + p^{q-1} \pmod {pq}\]
Thus
\[w + 1 = q^{p-1} - p^{q-1} + 1  = q^{p-1} - p^{q-1} + q^{p-1} + p^{q-1} = 2q^{p-1} \pmod {pq}\]
Therefore when we calculate $z$ in step 6 we obtain $q^{p-1}$ and obviously $\gcd(q^{p-1}, pq) = q$, and thus we indeed recover q and p in step 7 as required, since we got to step 4 in $O(t(n))$ steps and 4 through 7 also take $O(t(n))$ steps, $A'$ is an algorithm as request.
Randomization is required in our algorithm as we must get a root that is differs from the root we know not only by sign. Since we have no knowledge of what root $A$ will return, and since we cant find another root by ourselves, we must hope $A$ returns a different root, by choosing $y$ randomly many times, the probability we will indeed find a root that is different not only by sign approaches $1$.
 


\section{•}
\section{•}
\section{•}
We construct a polytime algorithm $A'$ that on input $p, g,g^x$ does the following:
\begin{itemize}
\item Draw $x\in \mathbb{Z^*}_{p}$ uniformly.
\item Execute $A$ on $p, g, g^{x+y}$ (note that $g^{x+y} = g^x g^y$), set $z$ to be the result.
\item  If $g^z = g^{x+y}$ return $z - y$, else if this is the 700'th time return 0, else go to the first step.
\end{itemize}
First note that this algorithm is polynomial as it executes $A$ at most 700 times, and $A$ is polynomial.
For each iteration the probability of landing within the subset of $x's$ for which $A$ finds and inverse is $\frac{1}{1000}$ as the sum of a uniform random variable and a constant is uniform. Hence with probability $\frac{1}{1000}$ we obtain the correct z in the last step, note that
\[g^{z-y} = g^z g^{-y} = g^{x+y}  g^{-y} = g^x\]
Thus $z-y$ is a solution to the DL problem.
The last step in $A$ fails only if $x+y$  is not inside the set for which $A$ solves the DL problem, this probability is $\frac{1}{1000}$ because $x+y$ distributes uniformly over $\mathbb{Z^*}_{p}$.

Because $A'$ makes 700 tries before returning with a false result, the probability that $A'$ fails is the probability that $A$ fails at each attempt which is $ (1-\frac{1}{1000})^{700} < \frac{1}{2}$.
Thus $A'$ is an algorithm as requested.

\section{}
\subsection{•}
We construct a the decryption function:
\[
Dec(c_1, c_2) = 
\begin{cases}
	 1 &\mbox{if }  c_1^x = c_2 \\ 
	 0 &\mbox{else}
\end{cases} 
\]
If $b=0$, then $c_2 = h^y = g^{xy} = c_1^x $, thus if $b=0$ $Dec$ returns the correct result.\\
If $b=1$ then given $z$ there is exactly one value of $y$ for which $g^y = g^{zx}$ since $g$ is a multiplicative generator, if  $g^y = g^{zx}$ then $y = zx \pmod {p-1}$, the only case in which we decrypt a $1$ to $0$ is if $y = xz$ which happens with probability at most $\frac{2}{p - 1}$.
Thus correct and efficient decryption is possible except for a negligible probability.
\subsection{•}
Assume that this encryption scheme is not $\epsilon CPA$ secure, then there is a polynomial adversary $A$ that wins the adversarial indistinguishability test with probability $ > \frac{1}{2} + \epsilon$.
We construct a polynomial time adversary $A'$ that shows DDH is not hard:
Given input $(g^x,g^y,g^z)$ our algorithm does the following:
\begin{itemize}
\item Supply $A$ with $(p, g, g^x)$.
\item Get the two messages from $A$ assume WLOG $A$ replays with $m_0 = 0$, $m_1 = 1$ (if this is not the case we can construct an algorithm $B$ that is based on $A$ and wins with the same probability, since if the messages are in a different order $B$ can change the order and if both messages have the same value $A$ can only guess which bit was chosen as both will be encrypted to the same value and $B$ can supply us with two messages and also guess and win with the same probability).
\item Supply $A$ with $(g^y, g^z)$, if $A$ returns 1 return 0, else return 1.  
\end{itemize}
We shall now show that $A'$ distinguishes $(g^x,g^y,g^z)$ from $(g^x,g^y,g^{xy})$:
\begin{align*}
&\Pr_{x,y \leftarrow U_{\mathbb{Z^*}_{p}}, z= xy}(A'(g^x, g^y, g^{z}) = 1) - \Pr_{x,y,z \leftarrow U_{\mathbb{Z^*}_{p}}}(A'(g^x, g^y, g^z) = 1)\\
& = \Pr(A'(g^x, g^y, g^{z}) = 1| x,y \leftarrow U_{\mathbb{Z^*}_{p}}, z= xy) - \Pr(A'(g^x, g^y, g^z) = 1| x,y,z \leftarrow U_{\mathbb{Z^*}_{p}})\\
&= \Pr(A \mbox{  wins } | b = 1) - \Pr(A \mbox{  loses }| b = 0)\\
&= 2[\Pr(A \mbox{  wins } \cap b = 1) - \Pr(A \mbox{  loses }\cap b = 0)]\\
&= 2[\Pr(A \mbox{  wins } \cap b = 1) - \Pr(b = 0) + \Pr(A \mbox{  wins }\cap b = 0)]\\
&= 2[\Pr(A \mbox{  wins }) - \Pr(b = 0) ]\\
&\geq 2[\frac{1}{2} + \epsilon - \frac{1}{2}] = 2\epsilon
\end{align*}

Note that in our calculation we refer to the probability that $x,y,z$ are drawn uniformly or $z=xy$ (this is $b$ as defined in the adversarial indistinguishability test) , each case has probability $\frac{1}{2}$ as we are in a distinguisher setup and thus are supplied with a sample from each distribution with equal probability (otherwise the streams are distinguishable by always saying that the current input originated from the stream with higher probability to be sampled).
Thus $A'$ is a distinguisher as required.
\end{document}