\documentclass{article}
\usepackage{listings}
\usepackage{amsmath}
\usepackage{amsfonts}
\usepackage{blindtext}
\usepackage{geometry}
\geometry{verbose,tmargin=3cm,bmargin=3cm,lmargin=3cm,rmargin=3cm}
\title{Crypto - HW 4}
\author{Hagai Ben Yehuda, ID num: 305237000\\ Jonathan Bauch, ID num: 204761233}
\date{}
\renewcommand{\thesubsection}{\thesection.\alph{subsection}}
\interlinepenalty 10000
\begin{document}
  \maketitle

\section{} % 1
\lstinputlisting[language=Python, caption=Q1 Code]{Q1.sage}
\lstinputlisting[caption=Q1 Output]{Q1_output.txt}
\textbf{Explanation}: Any $a$ with an $i$ as defined in the question is a witness that $m$ is not prime. Define $b := a^{(m-1)/2^i}$.
b is square  root of $1$ $\pmod{m}$ because:
\[ b^2 = (a^{(m-1)/2^i})^2 = a^{(m-1)/2^{i-1}} \equiv 1 \pmod{m} \]
But $b \not \equiv \pm 1 \pmod{m}$, therefore the polynomial $(x-1)^2$ has more than $2$ roots in $\mathbb{Z}_m$, which implies that $\mathbb{Z}_m$ is not a field (otherwise it would be a contradiction to the fundamental theorem of algebra). Therefore $m$ is not a power of a prime number, and specifically, it's not a prime number. 

\section{} % 2
We construct a randomized algorithm $A'$ that operates on input $m = pq$ as follows:
\begin{enumerate}
\item Draw $y\in Z_{m}^{*}$ uniformly (we do that by drawing from $\{1,...,m-1\}$ and making sure $\gcd(y,m)$ is zero, if it isn't we can factor m using $y$). .
\item Execute $A$ on input $y^2 \pmod m $ and set $x$ to be its result (note that since $y^2 \pmod m $ is a quadratic residue we will get a number and not  "go catch a Stellagama stellio"). 
\item If $x = \pm y \pmod m$ and this step was executed less then $c$ times ($c$ being a constant positive integer that will affect the probability of success) go to step one if this step was executed $c$ times, return $0$.
\item Calculate $w = x  y^{-1} \pmod m$.
\item Set $k = w+1 \pmod m$ 
\item Set $z = \frac{k}{2}$.
\item set $q = gcd(z, m)$ and return $(q, \frac{m}{q})$
\end{enumerate} 
We shall now prove that $A'$ runs in $O(t(n))$ and finds a factorization for $m$ with probability $ 1- \frac{1}{2^c}$.
First note that m executes steps 1 through 3 at most $c$ time (from the restriction in step 3) and each steps takes $O(t(n))$ steps.
In addition for each execution $A'$ passes step $3$ with probability $\frac{1}{2}$, that is because $y^2$ has four roots in $Z_{m}^{*}$, and only two of them are $\pm y$, since $y$ was chosen uniformly, the probability that the root that $A$ returns for $y^2$ is $ \pm y$  is  $\frac{2}{4} = \frac{1}{2}$.
Now we prove that if $A'$ passes step 3 it returns a correct factorization.

From the CRT we can write $x = wy$ with 
\[w = a_1(q^{-1} \pmod p)q + a_2(p^{-1} \pmod q)p\]
and $a_i \in \{\pm 1\}$ (this is because as stated in the lecture, if $x$ is a root of $y$ then it can be written as $ly$ with $l$ being a root of $1$ in $m$), since we chose x such that $x \neq \pm y$, we know that $a_1 \neq a_2$.\\
Assume without loss of generality that $a_1 = 1$ and  $a_2 = -1$, thus we have ( $w$ is from step 4)
\[w = (q^{-1} \pmod p)q - (p^{-1} \pmod q)p \]
Note that from fermat's little theorem we have 
\[(q^{-1} \pmod p)= q^{p-2} + cp\]
\[(p^{-1} \pmod q)= p^{q-2} + rq\]
Thus 
\[w = q^{p-1} - p^{q-1} \pmod {pq}\]
Note that 
\[q^{p-1} + p^{q-1} = 1 \pmod p\]
and
\[q^{p-1} + p^{q-1}  = 1 \pmod q\]
Hence
\[q^{p-1} + p^{q-1} \pmod {pq}\]
Thus
\[w + 1 = q^{p-1} - p^{q-1} + 1  = q^{p-1} - p^{q-1} + q^{p-1} + p^{q-1} = 2q^{p-1} \pmod {pq}\]
Therefore when we calculate $z$ in step 6 we obtain $q^{p-1}$ and obviously $\gcd(q^{p-1}, pq) = q$, and thus we indeed recover q and p in step 7 as required, since we got to step 4 in $O(t(n))$ steps and 4 through 7 also take $O(t(n))$ steps, $A'$ is an algorithm as request.
Randomization is required in our algorithm as we must get a root that is differs from the root we know not only by sign. Since we have no knowledge of what root $A$ will return, and since we cant find another root by ourselves, we must hope $A$ returns a different root, by choosing $y$ randomly many times, the probability we will indeed find a root that is different not only by sign approaches $1$.
 

\section{} % 3
\lstinputlisting[language=Python, caption=Q3 Code]{Q3.sage}
\lstinputlisting[caption=Q3 Output]{Q3_output.txt}
\textbf{Explanation}: TODO

\section{} % 4

\section{} % 5

We will construct $B$ as follows: given number $z=g^{y}$
\begin{enumerate}
\setcounter{enumi}{-1}
\item if z == 1, return $y=0$
\item choose random $r\in\left[1,p-1\right]$
\item calculate $z^{r}=g^{yr}$, $r^{-1}$ (using iterated squaring, and xgcd)
\item run $A\left(z^{r}\right)$
\begin{itemize}
\item if it succeeded and returned $yr$, then return $y$ by multiplying by $r^{-1}$
\item otherwise, return to step 1
\item if failed 1000 times, quit
\end{itemize}
\end{enumerate}
\textbf{Correctness}: As shown in a previous exercise, if $r$ is
uniformly distributed on $\left[1,p-1\right]$ and $y\in\left[1,p-1\right]$
then $ry$ is uniformly distributed on that range. And indeed $y\in\left[1,p-1\right]$
because in step 0 we rule out $y=0$. Therefore there is a $\frac{1}{1000}$
probability that $yr$ is in the exponents that $A$ can successfully
find. Therefore there is a $\frac{1}{1000}$ that step $3$ succeeds.
We then get:
\[
\Pr\left[\mbox{\ensuremath{B} succeeds}\right]=1-\left(\frac{999}{1000}\right)^{1000}\approx0.63>0.5
\]
\textbf{Run time}: step 0 is $O(1)$, choosing $r$ is $O(n)$ when
$n=\log_{2}\left(p\right)$. Calculating $z^{r}$ by repeated squaring
is $O(n^{2})$, and xgcd is $O(n)$. $A$ is poly-time in $n$, therefore it runs in time
$O(n^{k})$ for some $k\ge1$. then we get that a single iteration
of B takes $O(n^{2}+n^{k})$. Since there is a constant number of
iterations, the total runtime is $O(n^{2}+n^{k})=O\left(max\left\{ n^{2},n^{k}\right\} \right)$.

\section{} % 6

\subsection{} % 6.a
\textbf{Decryption}: Given $ \left<c_1, c_2\right> $:
\begin{itemize}
\item Compute $ c_1^x $
\item If $ c_1^x = c_2 $ return 0,
\item Otherwise return 1.
\end{itemize}
\textbf{Run-Time}: it is clear that this method is efficient, as it requires one exponentiation (which can be done efficiently by iterated squaring). \\
\textbf{Correctness}:
\begin{itemize}
\item  If the encrypted message is $ b = 0 $, then $ c_1^x = \left(g^y\right)^x = \left(g^y\right)^x = h^y = c_2 $. Therefore the decryption will succeed with probability 1.
\item If the encrypted message is $ b = 1 $, then $ c_1^x = g^{xy} $, $ c_2 =g^z $. Since $ y $ is random and independent from $ z $, then so is $ yx $. This mean that $ \Pr[yx \equiv z \pmod{p}] = \frac{1}{p} $. Therefore $ \Pr[c_1^x \neq c_2] = 1 - \frac{1}{p} $, meaning that the decryption succeeds with some negligible error probability.
\end{itemize}

\subsection{} % 6.b

\section{} % 7
Let $i \leftarrow U_t$ be the random index $A_1$ chooses. $b \leftarrow U_{0,1}$. $c^i = E_{pk}(m_b^i)$. Denote $A(x)$ the answer of an adversary $A$, given a cipher $x$. We have:
\begin{align*}
&\frac{1}{2} + \varepsilon \\
\mbox{(\ensuremath{\varepsilon}-CPA secure)} \ge &\Pr\left[\mbox{\ensuremath{A_1} wins}\right] \\
\mbox{(by definition)} = &\Pr\left[A_1(c^i)=b\right] \\
\mbox{(total probability)} = &\frac{1}{2} \sum_{d \in \{0,1\}} \Pr\left[A_1(c^i)=d \mid b=d\right] \\
\mbox{(total probability)} = &\frac{1}{2t} \sum_{d \in \{0,1\}} \sum_{k=1}^t \Pr\left[A_1(c^i)=d \mid b=d \wedge i=k\right] \\
\mbox{(by definition)} =  &\frac{1}{2t} \sum_{d \in \{0,1\}} \sum_{k=1}^t \Pr\left[A_1(E_{pk}(m_d^k))=d \mid b=d \wedge i=k\right] \\
\mbox{(by definition)} =  &\frac{1}{2t} \sum_{d \in \{0,1\}} \sum_{k=1}^t \Pr\left[A_{mult}\left(E_{pk}(m_0^1, \ldots, m_0^{k-1}, m_d^k, m_1^{k+1}, \ldots, m_1^t)\right)=d\right] \\
\mbox{(sum reorder)} = &\frac{1}{2t} \Bigg{(}
	\overbrace{\sum_{d \in \{0,1\}} \Pr\left[A_{mult}\left(E_{pk}(m_d^1, \ldots, m_d^t)\right)=d\right]}^{=2 \cdot \Pr\left[\mbox{\ensuremath{A_{mult}} wins}\right]} \\
& + \sum_{k=1}^{t-1} \overbrace{\sum_{d \in \{0,1\}} \Pr\left[A_{mult}\left(E_{pk}(m_0^1, \ldots, m_0^k, m_1^{k+1}, \ldots, m_1^t)\right)=d\right]}^{=1}
 \Bigg{)} \\
\mbox{(simplification)} = &\frac{1}{2t} \bigg{(} 2 \cdot \Pr\left[\mbox{\ensuremath{A_{mult}} wins}\right] + t-1 \bigg{)} \\
\mbox{(simplification)} = &\frac{1}{t} \Pr\left[\mbox{\ensuremath{A_{mult}} wins}\right] + \frac{1}{2} - \frac{1}{2t}
\end{align*}

Therefore:
\begin{align*}
\Pr\left[\mbox{\ensuremath{A_{mult}} wins}\right] 
\le t(\frac{1}{2} + \varepsilon - \frac{1}{2} + \frac{1}{2t})
= \frac{1}{2} + t \cdot \varepsilon
= \frac{1}{2} + \varepsilon_t
\ \ \ \ \ \square
\end{align*}


\section{} % 8

\subsection{} % 8.a
First, note that if $p,q$ are $n$ bit numbers, then $m=pq$ has at most $2n=O(n)$ bits.
Computing $a^{2^t}$ using iterated squaring involves $t$ steps of (modular) squaring a number in the range $[0,m-1]$: \[
a_0=a, a_1=a_0^2=a^2, a_2=a_1^2=a^{2^2}, a_3=a_2^2=a^{2^3}, \ldots, a_t=a_{t-1}^2=a^{2^t}
\]
Therefore the number of modular multiplications of $O(n)$ bit numbers is exactly $t$.

\subsection{} % 8.b
Knowing the factorization of $m$ allows us to compute $\phi(m)=(p-1)(q-1)$. Then, by Euler's theorem we know that $a^{2^t} \equiv a^{2^t \mod{\phi(m)}} \pmod{m}$. Therefore to compute $a^{2^t} \pmod{m}$ we need to calculate: \[
a_0=a, a_1=a_0^2=a^2, \ldots, a_k=a_{k-1}^2=a^{2^k}
\]
where $k$ is the number of bits of $2^t \mod \phi(m)$.
afterwards we multiply the elements according to the binary representation of $\phi(m)$.
$\phi(m)$ can have no more than $2n$ bits, therefore we need to perform at most $k+k = 2k \le 4n$ modular multiplications.
Note that if $2^t < \phi(m)$ then we resort to the first method and perform exactly $t$ multiplication. \\
So to summarize, we perform no more than $\min \{ t, 4n \}$ multiplications.

\end{document}