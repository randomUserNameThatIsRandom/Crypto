\documentclass{article}
\usepackage{listings}
\usepackage{amsmath}
\usepackage{amsfonts}
\usepackage{blindtext}
\usepackage{geometry}
\geometry{verbose,tmargin=3cm,bmargin=3cm,lmargin=3cm,rmargin=3cm}
\title{Crypto - HW 4}
\author{Hagai Ben Yehuda, ID num: 305237000\\ Jonathan Bauch, ID num: 204761233}
\date{}
\renewcommand{\thesubsection}{\thesection.\alph{subsection}}
\interlinepenalty 10000
\begin{document}
  \maketitle

\section{} % 1

\section{} % 2

\section{} % 3

\section{} % 4

\section{} % 5

We will construct $B$ as follows: given number $z=g^{y}$
\begin{enumerate}
\setcounter{enumi}{-1}
\item if z == 1, return $y=0$
\item choose random $r\in\left[1,p-1\right]$
\item calculate $z^{r}=g^{yr}$, $r^{-1}$ (using iterated squaring, and xgcd)
\item run $A\left(z^{r}\right)$
\begin{itemize}
\item if it succeeded and returned $yr$, then return $y$ by multiplying by $r^{-1}$
\item otherwise, return to step 1
\item if failed 1000 times, quit
\end{itemize}
\end{enumerate}
\textbf{Correctness}: As shown in a previous exercise, if $r$ is
uniformly distributed on $\left[1,p-1\right]$ and $y\in\left[1,p-1\right]$
then $ry$ is uniformly distributed on that range. And indeed $y\in\left[1,p-1\right]$
because in step 0 we rule out $y=0$. Therefore there is a $\frac{1}{1000}$
probability that $yr$ is in the exponents that $A$ can successfully
find. Therefore there is a $\frac{1}{1000}$ that step $3$ succeeds.
We then get:
\[
\Pr\left[\mbox{\ensuremath{B} succeeds}\right]=1-\left(\frac{999}{1000}\right)^{1000}\approx0.63>0.5
\]
\textbf{Run time}: step 0 is $O(1)$, choosing $r$ is $O(n)$ when
$n=\log_{2}\left(p\right)$. Calculating $z^{r}$ by repeated squaring
is $O(n^{2})$, and xgcd is $O(n)$. $A$ is poly-time in $n$, therefore it runs in time
$O(n^{k})$ for some $k\ge1$. then we get that a single iteration
of B takes $O(n^{2}+n^{k})$. Since there is a constant number of
iterations, the total runtime is $O(n^{2}+n^{k})=O\left(max\left\{ n^{2},n^{k}\right\} \right)$.

\section{} % 6

\subsection{} % 6.a
\textbf{Decryption}: Given $ \left<c_1, c_2\right> $:
\begin{itemize}
\item Compute $ c_1^x $
\item If $ c_1^x = c_2 $ return 0,
\item Otherwise return 1.
\end{itemize}
\textbf{Run-Time}: it is clear that this method is efficient, as it requires one exponentiation (which can be done efficiently by iterated squaring). \\
\textbf{Correctness}:
\begin{itemize}
\item  If the encrypted message is $ b = 0 $, then $ c_1^x = \left(g^y\right)^x = \left(g^y\right)^x = h^y = c_2 $. Therefore the decryption will succeed with probability 1.
\item If the encrypted message is $ b = 1 $, then $ c_1^x = g^{xy} $, $ c_2 =g^z $. Since $ y $ is random and independent from $ z $, then so is $ yx $. This mean that $ \Pr[yx \equiv z \pmod{p}] = \frac{1}{p} $. Therefore $ \Pr[c_1^x \neq c_2] = 1 - \frac{1}{p} $, meaning that the decryption succeeds with some negligible error probability.
\end{itemize}

\subsection{} % 6.b

\section{} % 7

\section{} % 8

\subsection{} % 8.a

\subsection{} % 8.b

\end{document}